% !TeX spellcheck = pt_BR
%%%%%%%%%%%%%%%%%%%%%%%%%%%%%%%%%%%%%%%%%
% Modelo de Proposta de Trabalho de Conclusão de Curso do Colegiado de Sistemas de Informação, peterncente ao
% Departamento de Computação e Sistemas de Informação (DECSI) da Universidade Federal de Ouro Preto (UFOP)
%
% LaTeX Template
%
% Versão 1.0 (22/04/2016)
%
% Autor Original:
% João Pedro Santos de Moura
% (https://github.com/jpmoura)
%
% Licença:
% CC BY-NC-SA 3.0 (http://creativecommons.org/licenses/by-nc-sa/3.0/)
%
%%%%%%%%%%%%%%%%%%%%%%%%%%%%%%%%%%%%%%%%%

%----------------------------------------------------------------------------------------
%	PACOTES E CONFIGURAÇÃO
%----------------------------------------------------------------------------------------

\documentclass[11pt,a4paper]{decsi-cosi}

\usepackage[brazil]{babel}                   % Pacote de vocabulário/internacionalização
\usepackage[utf8]{inputenc}                  % Pacote para codificação do texto em UTF-8
\usepackage[T1]{fontenc}                     % Pacote para codificação da fonte em artefatos DVI, PS e PDF
\usepackage{cmap}                            % Pacore para mapear caracteres especiais no documento fonte
\usepackage{textcomp,marvosym}               % Pacote para caracteres adicionais
\usepackage[brazilian,hyperpageref]{backref} % Paginas com as citações na bibl
\usepackage[alf]{abntex2cite}                % Citações padrão ABNT

%----------------------------------------------------------------------------------------
%	COMANDOS
%----------------------------------------------------------------------------------------

% Opções de metainfo do artefato gerado
\hypersetup{
	bookmarks=true,                           % Abrir aba de marcações de seções/capítulos no Acrobat Reader 
	unicode=true,                             % Caracteres latinos nos marcadores
	pdftoolbar=true,                          % Exibe a barra de ferramentas do Acrobat
	pdfmenubar=true,                          % Exibe o menu do Acrobat
	pdffitwindow=false,                       % A janela não se ajusta ao tamanho da página
	pdfstartview={FitH},                      % Ajusta a largura da página com o tamanho da janela de visualização
	pdftitle={Modelo de Proposta de TCC},     % Título
	pdfauthor={João Pedro Santos de Moura},   % Autor
	pdfsubject={Modelo LaTeX do COSI (UFOP) de Proposta de TCC}, % Assunto
	pdfcreator={pdflatex},  				  % Sistema de criação do PDF
	pdfproducer={LaTeX com hyperref}, 		  % Sistema em que foi produzido o Produtor
	pdfkeywords={LaTeX, Universidade Federal de Ouro Preto, Proposta de Trabalho de Conclusão de Curso, Modelo, Departamento de Computação e Sistemas de Informação}, % Lista de palavras-chave separadas por vírgula
	pdfnewwindow=true,                        % Links abrem em uma nova janela
	colorlinks=false,                         % false: links envoltos em uma caixa colorida; true: texto dos links coloridos
	linkcolor=red,                            % Cor de links internos (change box color with linkbordercolor)
	citecolor=green,                          % Cor de links para bibliografia
	filecolor=magenta,                        % Cor de links para arquivos
	urlcolor=cyan                             % Cor de links externos (url)
}

%----------------------------------------------------------------------------------------
%	DOCUMENTO
%----------------------------------------------------------------------------------------

\begin{document}
	\begin{center}
		\textbf{\uppercase{Proposta de Orientação de TCC}}
	\end{center}
	
	%\textbf{Aluno:} João Pedro Santos de Moura \hspace*{1in}\textbf{Matrícula:} 11.1.8338
	\aluno{Nome do Aluno}{Matrícula do aluno}
	
	\orientador{Nome do orientador}
	
	% Comente ou apague o comando seguinte caso não possua coorientador
	\coorientador{Nome do coorientador, caso exista}
	
	\disciplina{Nome da disciplina, exemplo: Sistemas de informação}
	
	\titulo{Título provisório do trabalho de conclusão.}
	
	\area{Codigo}{De acordo com as áreas CAPES/CNPQ}
	
	\tema
	
	Um parágrafo descrevendo o tema.\\
	
	\objetivos
	
	Descrição dos objetivos gerais e específicos.\\
	
	\problema
	
	Identificação do problema a ser abordado no projeto.\\
	
	% Cada coluna equivale um mês, por exemplo, a primeira coluna equivale ao mês 1.
	\abreCronograma
		Primeira Atividade & X & X & X & X & X & X & X & X & X & X & X &  \\ \hline
		Segunda Atividade  &   & X & X &   &   &   &   &   &   &   &   &  \\ \hline
		...                &   &   &   & X & X & X &   &   &   &   &   &  \\ \hline
		Última Atividade   &   &   &   &   &   &   & X & X &   &   &   &  \\ \hline
	\fechaCronograma
	
	(Cronograma de atividades para 12 meses, no caso de TCC1, devendo ser ajustado para a conclusão em 06 meses, no caso de TCC2)
	
	\referencias
	\bibliography{referencias}
	\nocite{EIA649B,abntex2classe,abntex2modelo-artigo,masolo2010,memoir}
	
	% O paramêtro corresponde ao espaçamento para o elemento textual imediatamente anterior.
	% Ajuste conforme sua necessidade. Os valores então em polegadas.
	\assinaturaOrientador{0.7in}
	
	\assinaturaCoorientador{1in}
	
\end{document}